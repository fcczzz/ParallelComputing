\documentclass{ctexart}
\usepackage{babel}
\usepackage{anyfontsize}
\usepackage[letterpaper,top=2cm,bottom=2cm,left=3cm,right=3cm,marginparwidth=1.75cm]{geometry}
\usepackage{amsmath, amssymb}
\usepackage{graphicx}
\usepackage{floatrow}
\usepackage[colorlinks=true, allcolors=blue]{hyperref}
\usepackage{enumitem}
\usepackage{longtable}
\usepackage{listings}
\usepackage{xcolor}
\usepackage{dirtree}
\definecolor{teal}{RGB}{0,128,128}
\setlist[itemize]{noitemsep}
\lstset{
    language=C++,
    basicstyle=\ttfamily\footnotesize,
    keywordstyle=\color{blue},
    commentstyle=\color{gray},
    stringstyle=\color{red},
    numbers=left,
    numberstyle=\tiny,
    stepnumber=1,
    numbersep=5pt,
    backgroundcolor=\color{lightgray},
    tabsize=2,
    showspaces=false,
    showstringspaces=false,
    frame=single,
    breaklines=true,
    postbreak=\mbox{\textcolor{red}{$\hookrightarrow$}\space},
}

\lstdefinestyle{Python}{
    language        =   Python, % 语言选Python
    basicstyle      =   \zihao{-5}\ttfamily,
    numberstyle     =   \zihao{-5}\ttfamily,
    keywordstyle    =   \color{blue},
    keywordstyle    =   [2] \color{teal},
    stringstyle     =   \color{magenta},
    commentstyle    =   \color{red}\ttfamily,
    breaklines      =   true,   % 自动换行,建议不要写太长的行
    columns         =   fixed,  % 如果不加这一句,字间距就不固定,很丑,必须加
    basewidth       =   0.5em,
}

\title{\textbf{并行计算大作业课程报告}}
\author{方驰正 PB21000163}

\begin{document}
\begin{sloppypar}

\maketitle

% \tableofcontents
% \newpage

% \ctexset{abstractname=摘要}
% \begin{abstract}

% \end{abstract}

\section{实验目的}
在本次实验中,我们使用MNIST手写数字数据集,训练了一个mlp自动感知机,并使用其预测了测试集,将结果提交至kaggle上的手写数字识别项目。分别实现了cpu串行版本和gpu并行版本,并对gpu并行版本进行优化,最终达到了96.1\%的准确率,以及相比于cpu版本的8倍的加速比。

\section{实验环境}
\begin{itemize}
    \item 操作系统:WSL2 Ubuntu 22.04
    \item 编程语言:C++, cuda
    \item 编译器:g++ 11.4.0
    \item cuda版本:12.4
    \item GPU:NVIDIA GeForce RTX 4060 Laptop
\end{itemize}

\section{算法设计和实现}
以下为主要的文件目录结构:
\dirtree{%
.1 {/} .
.2 {/data} .
.3 {train.csv} .
.3 {test.csv} .
.2 {/output} .
.3 {submission.csv} .
.2 {main.cpp} .
.2 {mnist.h} . 
.2 {mnist.cpp} .
.2 {matrix.h} .
.2 {matrix.cpp} .
.2 {Makefile} .
}
可以看到,核心代码主要分为三个部分,分别是mnist.h和mnist.cpp,matrix.h和matrix.cpp,以及main.cpp。其中mnist.h和mnist.cpp主要负责数据的读取和预处理,matrix.h和matrix.cpp主要负责矩阵运算,main.cpp主要负责模型的训练和预测。下面我们分别介绍这三个部分的实现。

\subsection{mnist}
这一部分为数据的读取和预处理部分。我们使用了mnist数据集,其中train.csv为训练集,test.csv为测试集。

具体地,在mnist.h中的声明如下:
\begin{lstlisting}[language=C++]
struct mnist_data {
    unsigned char label;
    unsigned char a[28 * 28];
};
std::vector<mnist_data> input(std::string filename, bool is_train);
void output(std::string filename, std::vector<int> data);
\end{lstlisting}

可以看到,我们定义了一个mnist\_data结构体,其中包含了一个label和一个28*28的数组,表示该数据的标签和特征。input函数用于读取数据,output函数用于输出数据。
\subsection{matrix}
\begin{lstlisting}[language=C++]
struct Mat {
    int n, m;
    double *a;
    double &operator()(int i, int j) const;
    void random_init(int n, int m, double loc, double scale); //随机初始化
    void zero_init(int n, int m);
    Mat();
    Mat(Mat &&_);
    Mat(const Mat &_);
    Mat operator=(Mat &&_);
    Mat operator=(const Mat &_);
    ~Mat();
    Mat operator*(const Mat &_) const;
    Mat operator*(const double &_) const;
    Mat operator+(const Mat &_) const;
    Mat operator-(const Mat &_) const;
    Mat relu() const;                                         //激活函数
    Mat relu_() const;                                        //激活函数的导数
    Mat softmax() const;                                      // softmax函数
    Mat softmax_() const;                                     // softmax函数的导数
    Mat T() const;                                            //转置
    double sum() const;

    Mat mult(const Mat &_) const; //矩阵对应元素相乘
};

double Loss(const Mat &y, const Mat &y_hat);
double Accuracy(const Mat &y, const Mat &y_hat);

\end{lstlisting}
在这里,我们定义了一个Mat结构体,用来表示一个矩阵。同时,我们定义了一系列在mlp训练与预测中需要用到的函数,如矩阵乘法、矩阵加法、矩阵减法、激活函数、softmax函数、损失函数、准确率等。
\subsection{mlp}
这是整个模型的核心部分,我们使用了一个三层的mlp模型,分别为输入层、隐藏层和输出层。其中,输入层包含784个神经元,隐藏层包含256个神经元,输出层包含10个神经元。我们使用了relu作为激活函数,使用softmax作为输出层的激活函数。同时,我们使用梯度下降方法进行反向传播。在训练时,我们使用了交叉熵作为损失函数。主要相关代码如下:
\begin{lstlisting}[language=C++]
struct MLP {
    Mat W1, W2, b1, b2;
    double lr;

    void forward(const Mat &input, Mat &z1, Mat &a1,
                 Mat &z2, Mat &a2) {
        z1 = input * W1 + b1;
        a1 = z1.relu();
        z2 = a1 * W2 + b2;
        a2 = z2.softmax();
    }

    void backward(Mat &input, Mat &z1, Mat &a1, Mat &z2,
                  Mat &a2, Mat &label) {
        Mat delta2 = (a2 - label).mult(z2.softmax_());
        Mat delta1 = (delta2 * W2.T()).mult(z1.relu_());

        Mat dW2 = a1.T() * delta2;
        Mat dW1 = input.T() * delta1;

        W1 = W1 - dW1 * lr;
        W2 = W2 - dW2 * lr;
        b1 = b1 - delta1 * lr;
        b2 = b2 - delta2 * lr;
    }

    std::pair<double, double>
    step(Mat &input, Mat &label,
         bool train = true) { 
        Mat z1, a1, z2, a2;
        forward(input, z1, a1, z2, a2);

        double loss = Loss(a2, label);
        double accuracy = Accuracy(a2, label);

        if (train) backward(input, z1, a1, z2, a2, label);
        return std::make_pair(loss, accuracy);
    }
} net;
\end{lstlisting}
可以看到,我们定义了一个MLP结构体,其中包含了W1、W2、b1、b2和lr等参数。forward函数用于前向传播,backward函数用于反向传播,step函数用于一次训练迭代。在训练时,我们使用了随机梯度下降法。
\begin{lstlisting}[language=C++]
void train(std::vector<mnist_data> &data, int epoch) {
    net.init(784, 256, 10, 0.01);
    ans = net;
    double best_accuracy = 0;
    double t = clock();
    for (int i = 0; i < epoch; i++) {
        std::vector<double> loss, accuracy;
        int t = 0;
        for (auto &d : data) {
            Mat input, label;
            input.zero_init(1, 784);
            for (int i = 0; i < 784; i++)
                input.a[i] = d.a[i] / 255.0;

            label.zero_init(1, 10);
            label.a[d.label] = 1;
            auto res = net.step(input, label);
            loss.push_back(res.first);
            accuracy.push_back(res.second);
        }
        double average_accuracy =
            std::accumulate(accuracy.begin(),
                            accuracy.end(), 0.0)
            / accuracy.size();
        if (average_accuracy > best_accuracy) {
            best_accuracy = average_accuracy;
            ans = net;
        }
    }
}

std::vector<int> test(std::vector<mnist_data> &data) {
    std::vector<int> res;
    for (auto &d : data) {
        Mat input;
        input.zero_init(1, 784);
        for (int i = 0; i < 784; i++)
            input.a[i] = d.a[i] / 255.0;

        Mat z1, a1, z2, a2;
        ans.forward(input, z1, a1, z2, a2);

        int predict = 0;
        for (int i = 0; i < 10; i++) {
            if (a2.a[i] > a2.a[predict]) predict = i;
        }
        res.push_back(predict);
    }
    return res;
}
\end{lstlisting}
如上为训练以及测试部分的代码。

可以看到,在训练时,我们将数据归一化到[0, 1]之间,并进行若干个epoch的训练。取平均准确率最高的模型作为最终的模型。为了简单起见,我们在测试时没有使用交叉验证,而是直接使用了训练集的数据进行测试。

在测试时,我们将数据归一化到[0, 1]之间,然后使用训练好的模型进行预测。最终将预测结果输出到submission.csv文件中。

\subsection{串行实验结果}
编译并运行后,输出如下:
\begin{lstlisting}
epoch:0 average_acc:0.861167 time used:128.734s
epoch:1 average_acc:0.925143 time used:257.458s
epoch:2 average_acc:0.941952 time used:384.517s
epoch:3 average_acc:0.952071 time used:510.324s
epoch:4 average_acc:0.959524 time used:635.946s
epoch:5 average_acc:0.964476 time used:770.660s
epoch:6 average_acc:0.968905 time used:899.398s
epoch:7 average_acc:0.972143 time used:1024.27s
epoch:8 average_acc:0.975429 time used:1150.56s
epoch:9 average_acc:0.978310 time used:1279.45s
\end{lstlisting}
可以看到,经过10个epoch的训练,最终的准确率为97.8\%,尚未过拟合。但是训练时间较长,达到了23分钟。将输出的测试数据提交至kaggle上,得到的准确率如下:
\begin{figure}[H]
    \centering
    \includegraphics[width=0.8\textwidth]{cpu_result.jpg}
    \caption{串行版本的kaggle提交结果}
\end{figure}
由于训练时间较长,我们尝试并行计算。
\section{并行实现和优化}
可以看到,串行版本的程序的运行时间还是比较长,因此我们尝试使用cuda,在gpu上进行并行计算。
由于主要的计算部分都在matrix.cpp中,因此我们重点的优化部分也在这里。这里,我们以矩阵乘法为例,介绍我们的优化方法。
\begin{lstlisting}[language=C++]
void Mat::zero_init(int N, int M) {
    n = N, m = M;
    a = new double[n * m];
    memset(a, 0, sizeof(double) * n * m);
}
Mat::~Mat() {
    if (a) delete[] a;
}
Mat Mat::operator*(const double &_) const {
    Mat res;
    res.zero_init(n, m);
    for (int i = 0; i < n; i++)
        for (int j = 0; j < m; j++)
            res(i, j) = (*this)(i, j) * _;
    return res;
}
\end{lstlisting}
这是最原始的矩阵乘法的实现,我们可以看到,我们首先在栈空间上申请了一个新的矩阵res,对其进行矩阵乘法运算,返回后调用析构函数释放内存。我们使用cuda优化后的代码如下:
\begin{lstlisting}[language=C++]
void Mat::zero_init(int N, int M) {
    n = N, m = M;
    cudaMalloc(&a, sizeof(double) * n * m);
    cudaMemset(a, 0, sizeof(double) * n * m);
}
Mat::~Mat() {
    if (a) cudaFree(a);
}
__global__ void mult_mat_kernel(double *a, double *b,
                                double *c, int n, int m,
                                int k) {
    int i = blockIdx.x * blockDim.x + threadIdx.x;
    if (i < n * k) {
        int x = i / k, y = i % k;
        c[i] = 0;
        for (int j = 0; j < m; j++)
            c[i] += a[x * m + j] * b[j * k + y];
    }
}
Mat Mat::operator*(const Mat &_) const {
    assert(m == _.n);
    Mat res;
    res.zero_init(n, _.m);
    mult_mat_kernel<<<(n * _.m + 1023) / 1024, 1024>>>(
        a, _.a, res.a, n, m, _.m);
    return res;
}
\end{lstlisting}
可以看到,我们仅仅是将核心的计算部分改为了cuda的核函数。将这份代码编译训练1个epoch后,得到如下结果:
\begin{lstlisting}
epoch:0 average_acc:0.856976 time used:71.9752s
\end{lstlisting}
可以看到,相比于cpu版本,速度提升了将近一倍。然而我们发现,cuda版本的速度并没有达到我们的预期,因此我们尝试进一步优化。使用NVIDIA的nsight进行性能分析,得到如下结果:

\section{实验结果}



\ctexset{bibname=参考资料}
\begin{thebibliography}{100}
      \bibitem{ref1}\href{https://www.kaggle.com/code/ambrosm/pss3e20-eda-which-makes-sense}{PSS3E20 EDA which makes sense}
      \bibitem{ref2}\href{https://www.kaggle.com/code/kacperrabczewski/rwanda-co2-step-by-step-guide}{Rwanda CO2: Step by step guide}
      \bibitem{ref3}\href{https://www.kaggle.com/code/yaaangzhou/pg-s3-e20-eda-modeling}{[PG S3 E20] EDA + Modeling}
      \bibitem{ref4}\href{https://www.kaggle.com/code/dmitryuarov/ps3e20-rwanda-emission-advanced-fe-20-88}{PS3E20 | Rwanda emission | Advanced FE | 20.88}
\end{thebibliography}

\end{sloppypar}
\end{document}
